%!TEX root = ./main.tex
This is the version to include data analysis.
New big version.
%
%
\begin{table}[htb]
	\begin{center}
		\begin{tabular}{cllcl}
			\toprule
				Symbol
				&\multicolumn{1}{c}{Meaning} 
				&Reference
				& Range 
				& Units
			\\
			\midrule
				$b$
				& Biting rate
				&\cite{YasunoM1990}
				&[1.48 , 4.77] 
				&$\si{meals \per day}$ 
			\\
				$\Lambda_H$
				& Human birth rate
				&
				& $\mu_H \cdot N_H$
				& $\si{day^{-1}}$
			\\
				$\Lambda_V$	
				& Vector birth rate
				&
				& $\mu_V \cdot N_V$
				&$\si{day^{-1}}$
			\\
				$\mu_M$
				&	vector mortality rate
				&\cite{Muir1998,YANG2009} 
				& [\num{0.036}, \num{0.109}] 
				& $\si{day^{-1}}$
			\\
				$\mu_H$
				& Human mortality
				& ---------
				&\num{3.9139E-05} 
				&$\si{day^{-1}}$
			\\
				$N_H$
				& Whole human population
				& ---------
				&\num{700000}
				& ---------
			\\
				$\gamma^{-1}$
				& Average time which a
				&
				&
				&
			\\
				&mosquito acquires a
				& ---------
				& ---------
				& $\si{day}$
			\\
				& high virus load
				&
				&
				&
			\\
				$\beta_H$ 
				& Human infection
				&
				&
			\\
				& probability  by vectors
				&
				& (\num{0}, \num{0.05}] 
				& ---------
			\\
				$\beta_M$
				& Vector infection 
				&
				&
				&
			\\
				& probability by humans
				&
				& (\num{0}, \num{0.05}] 
				& ---------
			\\
				$p_c$
				& Probability of acquire  
					 
			\\
				
<<<<<<< HEAD
				& classic dengue,
					given a  
			\\
				& load virus mosquito bit.
				&
			\\
				$\alpha_{a}$ 
				& Mean recover rate 
			\\
				& from Asymptomatic 
				&
				&
				&
			\\
				& Dengue
			\\
				$\alpha_{c}$ 
				& Mean recover rate \\
				& from Classic Dengue
				&
				&
			\\
				$\alpha_{h}$
				& Mean recover rate 
				&
				&
			\\
				& from Hemorrhagic 
			\\
				& Dengue
=======
			$\beta_H$ &	Human infection \\
            	&		probability  by vectors
				&	& (\num{0}, \num{0.05}] & ---	\\
			$\beta_M$	& Vector infection \\
            		& probability by humans
				&	& (\num{0}, \num{0.05}] & \cite{Feng1997a}\\
			$\alpha_{c}$ & Mean recover rate \\
            	&	from Classic Dengue
				&	&	
			\\
			$\alpha_{h}$	& Mean recover rate \\
            				& from Hemorrhagic  Dengue
			$\sigma$			& Suceptibility two stain 2
>>>>>>> 7b2fffd8fcc53b55ea3fe9907bd9d0336fc57245
			\\
			\bottomrule
		\end{tabular}
	\end{center}
	\caption{Parameter description}
\end{table}
\section*{New proposals Oct-23-2017}

\section{Virus load model.}
\begin{equation}
	\begin{aligned}
		\frac{dM_S}{dt}&= 
			\Lambda_M - \frac{\beta_Mb}{N_H}
			(H_a+H_c+H_h)M_S-\mu_MM_S
		\\
		\frac{dM_1}{dt}
			&=
			\frac{\beta_Mb}{N_H}
			(H_a + H_c + H_h) M_S 
			-
			(\gamma_1 + \mu_M) M_1
		\\
		\frac{dM_2}{dt}&=
			\gamma_1M_1-\mu_MM_2
		\\
		\frac{dH_S}{dt}&=
			\Lambda_H
			-\frac{\beta_H b}{N_H}(M_1 + M_2)H_S
			-\mu_HH_S
	\\
		\frac{dH_a}{dt}&=
			\frac{\beta_H b}{N_H} M_1 H_S
			-
			(\alpha_a + \mu_H) H_a
	\\
		\frac{dH_c}{dt}&=
			\frac{\beta_H b p_c}{N_H} M_2 H_S
			-(\alpha_c + \mu_H) H_c
	\\
		\frac{dH_h}{dt}&=
			\frac{\beta_H b(1-p_c)}{N_H}
			M_2 H_S
		-
		(\alpha_h + \mu_H) H_h
	\end{aligned}
\end{equation}
Taking the infection forces as
\begin{align*}
	A_a &:= 
		\frac{\beta_Mb}{N_H} H_a, 
	&
	A_c &:=
		\frac{\beta_Mb}{N_H}H_c ,
	&
	A_h &:=
		\frac{\beta_Mb}{N_H}H_h,
	\\
	B_1&:=
		\frac{\beta_Hb}{N_H}M_1,
	&
	B_2&:=
		\frac{\beta_Hbc_2}{N_H}M_2, 
	&
	B_3&:=
		\frac{\beta_Hb(1-c_2)}{N_H}M_2,
\end{align*}
yields
\begin{equation}
	\begin{aligned}
		\frac{dM_S}{dt}
			&= \Lambda_M 
			-(A_a + A_c + A_h) M_S 
			- \mu_M M_S
		\\
		\frac{dM_1}{dt}
			&= (A_a + A_c + A_h) M_S 
			-(\gamma_1 + \mu_M) M_1
		\\
		\frac{dM_2}{dt}
			&=\gamma_1M_1 - \mu_MM_2
		\\
		\frac{dH_S}{dt}
			&=\Lambda_H - 
			\frac{\beta_Hb}{N_H}
				(M_1 + M_2)
				H_S-\mu_HH_S
		\\
		\frac{d H_a}{dt}
			&=B_1H_S - (\alpha_a + \mu_H) H_a
		\\
		\frac{dH_c}{dt}
			&=B_2H_S
			-(\alpha_c+\mu_H) H_c
		\\
		\frac{dH_h}{dt}
			&=B_3H_S
			-(\alpha_h+\mu_H) H_h
	\end{aligned}
\end{equation}
%
%
%
\subsection{Model description and parameters meaning}
	
\subsection{Discussion of $\mathcal{R}_0$}
	Using the next generation operator we gets
	
	\begin{equation}
		\mathcal{R}_0 =
			\sqrt{
				\frac{\beta_M \beta_H b^2 N_M}{(\gamma_1 + \mu_M) N_H }
				\left(
					\frac{(1-c)\gamma_1}{\mu_M (\alpha_h + \mu_H)}+
					\frac{c \gamma_1}{\mu_M (\alpha_c + \mu_H)}+
					\frac{1}{ \alpha_a + \mu_H}
				\right)
			}
	\end{equation}
<<<<<<< HEAD
\section{A two strain model.}
%%
\begin{equation}
	\begin{aligned}
		\frac{dM_S}{dt}&=
			\Lambda_M
			-\frac{\beta_Mb}{N_H}
			(I_1 + I_2 + Y_{-1} + Y_{-2})M_S
			-\mu_MM_S
		\\
		\frac{dM_1}{dt}&=
			\frac{\beta_Mb}{N_H}
			(I_1 + Y_{-2}) M_S - \mu_M M_1
		\\
		\frac{dM_2}{dt}&=
			\frac{\beta_Mb}{N_H}
			(I_2 + Y_{-1}) M_S - \mu_M M_2
		\\
		\frac{dI_S}{dt}&=
			\Lambda_S 
			- \frac{ \beta_H b }{N_H}
				(M_1 + M_2)I_S - \mu_H I_S
		\\
		\frac{dI_1}{dt}&=
			\frac{\beta_Hb}{N_H}
			M_1 I_S - (\alpha_c + \mu_H) I_1
		\\
		\frac{dI_2}{dt}&=
			\frac{\beta_Hb}{N_H} M_2 I_S 
			- ( \alpha_c + \mu_H ) I_2
		\\
		\frac{dS_{-1}}{dt}&=
			\Lambda_{-1}
			-\frac{ \beta_Hb }{N_H} M_2 S_{-1}
			-\mu_HS_{-1}
		\\
		\frac{dS_{-2}}{dt}&=
			\Lambda_{-2}
			-\frac{\beta_Hb}{N_H}M_1S_{-2}
			-\mu_HS_{-2}
		\\
		\frac{dY_{-1}}{dt}&=
			\frac{\beta_Hb}{N_H}
			M_2 S_{-1}
			- (\alpha_h + \mu_H) Y_{-1}
		\\
		\frac{dY_{-2}}{dt}&=
			\frac{\beta_Hb}{N_H}
			M_1 S_{-2}
			-( \alpha_h + \mu_H) Y_{-2}
	\end{aligned}
\end{equation}
Defining the infection forces as
\begin{equation}
	\begin{aligned}
		A_1&:=
			\frac{\beta_Mb}{N_H} I_1,
		&
		A_2&:=
			\frac{\beta_Mb}{N_H} I_2, 
		&
		A_3&:=
			\frac{\beta_Mb}{N_H} Y_{-1},
		&
		A_4&:=
			\frac{\beta_Mb}{N_H} Y_{-2},
		\\
		B_1&:=
			\frac{\beta_Hb}{N_H} M_1, 
		&
		B_2&:=
			\frac{\beta_Hb}{N_H} M_2,
	\end{aligned}
\end{equation}
we rewrite the model \eqref{*} as
\begin{equation}
\begin{aligned}
	\frac{dM_S}{dt}&= 
		\Lambda_M
		-(A_1 + A_2 + A_3 + A_4) M_S
		-\mu_M M_S
	\\
	\frac{dM_1}{dt}&=
		(A_1+A_4)M_S - \mu_M M_1
	\\
	\frac{dM_2}{dt}&=
		(A_2+A_3)M_S-\mu_MM_2
	\\
	\frac{dI_S}{dt}&=
		\Lambda_S - (B_1 + B_2)I_S
		-\mu_HI_S
	\\
	\frac{dI_1}{dt}&=
		B_1 I_S
		-(\alpha_c + \mu_H) I_1
	\\
	\frac{dI_2}{dt}&=
		B_2 I_S 
		-(\alpha_c + \mu_H)I_2
	\\
	\frac{dS_{-1}}{dt}&=
		\Lambda_{-1}-B_2S_{-1}-\mu_HS_{-1}
	\\
	\frac{dS_{-2}}{dt}&=
		\Lambda_{-2}-B_1S_{-2}-\mu_HS_{-2}
	\\
	\frac{dY_{-1}}{dt}&=
		B_2S_{-1}
		-(\alpha_h + \mu_H)Y_{-1}
	\\
	\frac{dY_{-2}}{dt}&=
		B_1 S_{-2} 
		- (\alpha_h + \mu_H) Y_{-2}
	\\
\end{aligned}
\end{equation}
%
%
$I_i$ is the infected individual class with serotype $i$; $S_{-i}$ are the 
individuals who were infected with serotype $i$ in previous years and for the 
current year are susceptible to a serotype different from $i$; $Y_{-i}$ is the 
class of reinfected individuals with serotype different from $i$.

	The disease free equilibrium results
$$
	FDE^*=
	\left(
		\frac{\Lambda_M}{\mu_M},
		0,
		0,
		\frac{\Lambda_S}{\mu_H},
		0,
		\frac{\Lambda_{-1}}{\mu_H},
		0,
		\frac{\Lambda_{-2}}{\mu_H},
		0
	\right).
$$
Next, using the next generation operator method (citar a velasco) we obtain
the basic reproductive number
%
\begin{equation}
	R_0=
	\max\
	\left\{ 
		\sqrt{
			\frac{\beta_H\beta_Mb^2N_M}{\mu_M\mu_HN_H^2}
			\left(
				\frac{\Lambda_{-1}}{\alpha_h+\mu_H}
				+
				\frac{\Lambda_H}{\alpha_c+\mu_H} 
			\right)
		},
		\sqrt{
			\frac{\beta_H\beta_Mb^2N_M}{\mu_M\mu_HN_H^2}
			\left(
				\frac{\Lambda_{-2}}{\alpha_h+\mu_H}
				+
				\frac{\Lambda_H}{\alpha_c+\mu_H}
			\right)
		}
	\right\}.
\end{equation}
%
=======



%%
>>>>>>> 7b2fffd8fcc53b55ea3fe9907bd9d0336fc57245
