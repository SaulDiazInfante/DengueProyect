%!TEX root = main.tex
\paragraph{In the Discussion-Conclusions it is essential to:}

\begin{itemize}
    \item
        be clear what YOU did and what other authors have done
    \item
        highlight your UNIQUE contribution
    
    \item
        discuss LIMITATIONS of your findings state what the
         applications and implications of your research are
\end{itemize}

The Discussion should answer the following questions, and possibly in the following order. 
You can thus use the answers to structure your Discussion. This gives you
a relatively easy template to follow.


\noindent One of the main limitations of our model is its usefulness for only one year dynamics. 
Our model was written with two purposes. The first was to keep track of two different 

The main focus on our model was to establish the causes of having DHF in the population based on
having DENV-2 as the second infection 

\begin{enumerate}
    \item
        What are my most important findings?
    \item
        Do these findings support what I set out to demonstrate at the beginning of the
        paper?
    \item
        How do my findings compare with what others have found? How consistent are
        they?
    \item
        What is my personal interpretation of my findings?
    \item
        What other possible interpretations are there?
    \item
        What are the limitations of my study? What other factors could have influenced
        my findings? Have I reported everything that could make my fi ndings invalid?
    \item
        Do any of the interpretations reveal a possible flaw (i.e. defect, error) in my
        experiment?
    \item
        Do my interpretations contribute some new understanding of the problem that I
        have investigated? In which case do they suggest a shortcoming in, or an
        advance on, the work of others?
    \item
        What external validity do my findings have? How could my findings be 
        generalized to other areas?
    \item
        What possible implications or applications do my fi ndings have? What support
        can I give for such implications?
    \item
        What further research would be needed to explain the issues raised by my 
        findings? Will I do this research myself or do I want to throw it open to the
        community?
\end{enumerate}
    A JBM structure for discussion:
	\paragraph{Statement of principal findings}
	\begin{enumerate}
		\item
			Remind readers of your goals, preferably in a single sentence:
			One of the main goals of this experiment was to attempt
			 to find a way to predict who shows more task persistence.
		\item
			Refer back to the questions (hypotheses, predictions etc.) 
			that you posed in your Introduction:
			These results both negate and support some of the hypotheses.
			It was predicted that greater perfectionism scores would result
			in greater task persistence, but this turned out not to be
			the case.
		\item
			Refer back to papers you cited in your Review of the Literature:
			Previous studies conflict with the data presented in the Results: 
			it was more common for any type of feedback to impact participants 
			than no feedback (Shanab et al., 1981; Elawar \&  Corno, 1985).
		\item
			Briefly restate the most important points from your Results:
			While not all of the results were significant, 
			the overall direction of results showed trends 
			that could be helpful to learning about who is more 
			likely to persist and what could influence persistence.
	\end{enumerate}
    \paragraph{Statement of principal findings}
        The central hypotheses of this manuscript were:
        \begin{itemize}
            \item
                strain DENV-II circulated in the 2010 Hermosillo 
            \item
                Dengue outbreak antibody-Dependent Enhancement hypothesis
                could explain the HDF high incidence 
            \item
                unconfirmed cases represent almost 95 \% of the total
                incidence.
        \end{itemize}
    With these hypotheses, we formulated a model that describes
    the incidence dynamics of HDF cases of the 2010 Hemosillo outbreak.
    Our parameters estimation does not reject these hypothesis and provide
    statistical evidence that the concerning reproductive number was 2.47.
    


    \paragraph{Strengths and weaknesses of the study}
        We believe this is the first attempt to fit the exponential 
    growth of the incidence according to the Hemorragic Dengue Disease. 
    Further, the strucutre of our R0 calcuoation results be 
    the gemetric mean of to expresion that are closed related 
    with the parameters transmition according to the Hemorragic.  

    
        However, our formulation only considers an outbreak of one season.
    Thus classical asymptotic analysis has mathematical consistency
    but lacks biological meaning.  In this line, we are preparing a
    version that achieves both \textemdash proper data fitting and consistent
    asymptotic behavior. We believe that adding a class of immunity
    according to the particular strain would be an option.

    \paragraph{Strengths and weaknesses in relation 
        to other studies, discussing important
        differences in results}
        
        We apply the idea of an index of strain susceptibility as in [1,2,3] 
    to modulates the propensity to acquire Dengue due to a given serotype.  Thus,
    combining the hypothesis of unreported incidence [3,4,5], we achieve
    proper data fitting.

        Our contribution describes scenarios wherein past Dengue outbreaks
    detected a dominant strain. Then decision-makers could project new
    dengue strain invasion scenarios.

        The cause of HDF is still unclear; for example, in [6] reports that
    the concentration of the virus explains the severity of Dengue. In
    this direction, Gomez et al. report a model that considers a severity
    stratification according to the mosquito load virulence.

        In short, we believe that early serotype identification in any Dengue
    outbreak would be essential infromation to make epidemiological decisions.
    %
    \paragraph{Meaning of the study: possible explanations 
    and implications for clinicians
    and policymakers}
    \paragraph{Unanswered questions and future research}


        The central hypotheses of this manuscript are: 
        
    \begin{itemize}
        \item
            strain DENV-II
            circulated in the 2010 Hermosillo Dengue outbreak 
            antibody-Dependent
        \item
            enhancement hypothesis could explain the HDF high incidence
        \item
            unconfirmed cases represent almost \num{95 \%} 
            of the total incidence,        
    \end{itemize}
    with these hypotheses, we formulated a model that 
    describes the incidence dynamics of HDF cases of the 
    \num{2010} Hemosillo outbreak. Our parameter estimation does not reject
    this hypothesis and  provides statistical evidence that the reproductive 
    number was \num{2.45}.

    \paragraph{Strengths and weaknesses of the study} 
    Our R0 estimation is consistent with other outbreak reports [*].
    To the best of our knowledge, this is the first attempt to fit the 
    two exponential growth curves of the incidence according to Dengue 
    Disease severity. However, our formulation only considers an outbreak
    of one season. Thus classical asymptotic analysis has mathematical 
    consistency but lacks biological meaning.  In this line, we are preparing
    a version that achieves both\textemdash proper data fitting and consistent
    asymptotic behavior. We believe that adding a class of immunity according
    to the particular strain would be an option.

        Our contribution describes scenarios wherein past Dengue
    outbreaks detected a dominant  strain. Then decision-makers could
    project new dengue strain invasion scenarios.
     
        Applying an index of strain susceptibility as in  [Kooi2014,
    Zheng2018, Nuraini2007, Feng1997a] to modulates the propensity to
    acquire Dengue due to a given serotype,  we describe DHF transmission
    with a certain probability.  Thus, combining this idea with the
    hypothesis of unreported incidence [Li2013b,4,5], we achieve proper
    data fitting. 

        The cause of DHF is still unclear; for example, in ESTEVA2015a reports
    that the concentration of the virus explains the severity of Dengue.
    In this direction, Gomez et al. provide a model that considers a
    severity stratification according to the mosquito load virulence.

        In short, we believe that early serotype identification in any Dengue
    outbreak would be essential to  make epidemiological decisions.
