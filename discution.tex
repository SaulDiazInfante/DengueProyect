%!TEX root = main.tex
\paragraph{In the Discussion-Conclusions it is essential to:}

\begin{itemize}
	\item
		be clear what YOU did and what other authors have done
	\item
		highlight your UNIQUE contribution
	
	\item
		discuss LIMITATIONS of your findings state what the
		 applications and implications of your research are
\end{itemize}

The Discussion should answer the following questions, and possibly in the follow-
ing order. You can thus use the answers to structure your Discussion. This gives you
a relatively easy template to follow.


\begin{enumerate}
	\item
		What are my most important findings?
	\item
		Do these findings support what I set out to demonstrate at the beginning of the
		paper?
	\item
		How do my findings compare with what others have found? How consistent are
		they?
	\item
		What is my personal interpretation of my findings?
	\item
		What other possible interpretations are there?
	\item
		What are the limitations of my study? What other factors could have influenced
		my findings? Have I reported everything that could make my fi ndings invalid?
	\item
		Do any of the interpretations reveal a possible flaw (i.e. defect, error) in my
		experiment?
	\item
		Do my interpretations contribute some new understanding of the problem that I
		have investigated? In which case do they suggest a shortcoming in, or an
		advance on, the work of others?
	\item
		What external validity do my findings have? How could my findings be 
		generalized to other areas?
	\item
		What possible implications or applications do my fi ndings have? What support
		can I give for such implications?
	\item
		What further research would be needed to explain the issues raised by my 
		findings? Will I do this research myself or do I want to throw it open to the
		community?
\end{enumerate}
A JBM structure for discussion:
	\paragraph{Statement of principal findings}
	\begin{enumerate}
		\item
			Remind readers of your goals, preferably in a single sentence:
			One of the main goals of this experiment was to attempt
			 to find a way to predict who shows more task persistence.
		\item
			Refer back to the questions (hypotheses, predictions etc.) 
			that you posed in your Introduction:
			These results both negate and support some of the hypotheses.
			It was predicted that greater perfectionism scores would result
			in greater task persistence, but this turned out not to be
			the case.
		\item
			Refer back to papers you cited in your Review of the Literature:
			Previous studies conflict with the data presented in the Results: 
			it was more common for any type of feedback to impact participants 
			than no feedback (Shanab et al., 1981; Elawar \&  Corno, 1985).
		\item
			Briefly restate the most important points from your Results:
			While not all of the results were significant, 
			the overall direction of results showed trends 
			that could be helpful to learning about who is more 
			likely to persist and what could influence persistence.
	\end{enumerate}

	\paragraph{Strengths and weaknesses of the study}
	\paragraph{Strengths and weaknesses in relation 
		to other studies, discussing important
		differences in results}
	\paragraph{Meaning of the study: possible explanations 
	and implications for clinicians
	and policymakers}
	\paragraph{Unanswered questions and future research}

\paragraph{Highligting our finding}