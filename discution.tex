%!TEX root = main.tex
    \paragraph{Statement of principal findings}
        We estimate that $R_{0c}>1$ and  $R_{0h}<1$. Thus, for this outbreak, 
    the presence of DHF cases, cannot trigger new DHF cases 
    and in general, $R_{0h}<1$ would imply an exponential decay on the number 
    of infected individuals. However,
    the small DHF outbreak arises despite the value of $R_{0h}$ as
    there is an increase in infected mosquitoes of the serotype 2 due
    to the presence of the $S$ individuals, which initially is close
    to $N_S$. Therefore, DHF dynamics is a consequence of the
    intensity of the outbreak of DF, given by serotype 2.

        Our model was written following two purposes. To describe the
    evolution of DCF and DHF cases in Hermosillo for the 2010 outbreak and
    to evaluate the hypothesis that secondary infections with DENV-2
    serotype were the main responsible for DHF cases. In order to meet our
    needs, our  model is useful for one year dynamics only.    

        The central hypotheses of this manuscript were:
        \begin{itemize}
            \item
                strain DENV-II circulated in the 2010 Hermosillo 
            \item
                Dengue outbreak antibody-Dependent Enhancement hypothesis
                could explain the HDF high incidence 
            \item
                unconfirmed cases represent almost 95 \% of the total
                incidence.
        \end{itemize}
        With these hypotheses, we formulated a model that describes
    the incidence dynamics of HDF cases of the 2010 Hemosillo outbreak.
    Our parameters estimation does not reject these hypothesis and provide
    statistical evidence that the concerning reproductive number was \num{2.47}.
    
    \paragraph{Strengths and weaknesses of the study}
        We believe this is the first attempt to fit the exponential 
    growth of the incidence according to the Hemorragic Dengue Disease. 
    Further, the strucutre of our $\mathcal{R}_0$ calcuoation results be 
    the geometric mean of two expresion that are closed related 
    with the parameters transmition according to severity.  
    
        However, our formulation only considers an outbreak of one season.
    Thus classical asymptotic analysis has mathematical consistency
    but lacks biological meaning.  In this line, we are preparing a
    version that achieves both\textemdash proper data fitting and consistent
    asymptotic behavior. We believe that adding a class of immunity
    according to the particular strain would be an option.

    \paragraph{Strengths and weaknesses in relation 
        to other studies}
        
        Our contribution describes scenarios wherein past Dengue outbreaks
    detected a dominant strain. Then decision-makers could project new
    dengue strain invasion scenarios.

        The cause of HDF is still unclear; for example, in [6] reports that
    the concentration of the virus explains the severity of Dengue. In
    this direction, Gomez et al. report a model that considers a severity
    stratification according to the mosquito load virulence.

        In short, we believe that early serotype identification in any Dengue
    outbreak would be essential infromation to make epidemiological decisions.
    %
    \paragraph{Strengths and weaknesses of the study} 
        Our $\mathcal{R}_0$ estimation is consistent with other outbreak reports [*].
    To the best of our knowledge, this is the first attempt to fit the 
    two exponential growth curves of the incidence according to Dengue 
    Disease severity. However, our formulation only considers an outbreak
    of one season. Thus classical asymptotic analysis has mathematical 
    consistency but lacks biological meaning.  In this line, we are preparing
    a version that achieves both\textemdash proper data fitting and consistent
    asymptotic behavior. We believe that adding a class of immunity according
    to the particular strain would be an option.

        Our contribution describes scenarios wherein past Dengue
    outbreaks detected a dominant  strain. Then decision-makers could
    project new dengue strain invasion scenarios.
     
        Applying an index of strain susceptibility as in  [Kooi2014,
    Zheng2018, Nuraini2007, Feng1997a] to modulates the propensity to
    acquire Dengue due to a given serotype,  we describe DHF transmission
    with a certain probability.  Thus, combining this idea with the
    hypothesis of unreported incidence [Li2013b,4,5], we achieve proper
    data fitting. 

        The cause of DHF is still unclear; for example, in ESTEVA2015a reports
    that the concentration of the virus explains the severity of Dengue.
    In this direction, Gomez et al. provide a model that considers a
    severity stratification according to the mosquito load virulence.

        In short, we believe that early serotype identification in any Dengue
    outbreak would be essential to  make epidemiological decisions.
