%!TEX root = main.tex
\FloatBarrier
\paragraph{Statement of principal findings}
    We estimate that $R_{0c}>1$ and  $R_{0h}<1$. Thus DHF cases,
cannot sustain new DHF cases. Furhter, $R_{0h}<1$ imply an
exponential decay on the number of infected individuals. However,
since a small DHF outbreak arises despite the value of $R_{0h}$, we deduce
that this DHF dynamics is a consequence of the
intensity of the outbreak of DF, given by serotype 2.

    Our formulation describe the evolution of DCF and DHF
cases in Hermosillo for the \num{2010} outbreak and evaluate if ADE
hypotesis could explain the DHF dynamics.
The central hypotheses of this manuscript were:
    \begin{itemize}
        \item
            First time Hermosillo circulation of strain DENV-II ocurre
            in the 2010
        \item
            Dengue outbreak antibody-Dependent Enhancement hypothesis
            could explain the DHF high incidence
        \item
            unconfirmed cases represent almost 95 \% of the total
            incidence.
    \end{itemize}
    With these hypotheses, we obtain a model that describes
the incidence dynamics of DHF cases of the 2010 Hemosillo outbreak.
Our parameters estimation does not reject these hypothesis and provide
statistical evidence that the concerning reproductive number is in the
\num{95} \% confidence interval {[\num{1.81253}, \num{2.134538}]}.
%
%
\paragraph{Strengths and weaknesses of the study}
    Our $\mathcal{R}_0$ estimation is consistent with other outbreak reports
\cite{Khan2014}.
To the best of our knowledge, this is the first attempt to fit the
two exponential growth curves of the incidence according to Dengue
Disease severity.  Furter, our reproductive numeber $\mathcal{R}_0$
results be the geometric mean of two expresion that are closed related
with parameters transmition according to severity.

    However, our formulation only considers an outbreak
of one season. Thus classical asymptotic analysis has mathematical
consistency but lacks biological meaning.  In this line, we are preparing
a version that achieves both\textemdash proper data fitting and consistent
asymptotic behavior. We believe that adding a class of immunity according
to the particular strain would be an option.


     We take ideas from \citet{Zheng2018, Nuraini2007, Feng1997a} to modulate
the propensity of acquire Dengue due to a given serotype.
Our model describe DHF transmission with a certain probability.
\tinytodo{Mathematical Medicine and Biology:%
 A Journal of the IMA, dqz013,
\href{https://doi.org/10.1093/imammb/dqz013}{ref}}
Thus, combining this idea with the hypothesis of unreported
incidence \cite{Li2013b,Guzman2002}, we achieve proper
data fitting via maximum likelihood estimation.

The cause of DHF is still obscure. Since the reported cases of dengue patients
with hemorrhagic in its first infection, new hypotheses have arisen
\cite{Debast1993}.  However, the ADE hypothesis sill is the most widely
accepted to explain DHF.

We believe that early serotype identification in any Dengue outbreak would be
essential to make epidemiological decisions.
