%!TEX root = main.tex
State the objectives of the work and provide an adequate 
background, avoiding a detailed literature survey or a 
summary of the results.
\begin{itemize}
	\item
    	State of the art (literature review).
    \item	
    	Knowledge gap.
    \item	
    	Objective.
    \item
    	Methodology.
    \item	
    	Results.
    \item
    	Conclusions.
\end{itemize}

\section{Introduction.}\label{intro}
\paragraph{Contribution}
Our objective is to explain the dengue hemorrhagic outbreak that 
occurred in the city of Hermosillo, located in the state of Sonora in 2010.
    %
\paragraph{Severity}
\ac{DCF} 
\ac{DHF} description.
\paragraph{ADE hypothesis}
\ac{DENV-1}, \ac{DENV-2}
reinfection as cause of hemorrhagic.
\paragraph{Background}

	Dengue fever is one of the diseases that has advanced in the 
world due to climate change. One of the main strategies to control dengue has been the control of mosquito population.

	As mentioned by Ernst et. al. (\cite{Ernst2016}) in Nogales there 
is much more propagating conditions than in Hermsosillo. However, 
due to temperature, Dengue has not strike more Nogales. Can we 
implement a second variable (related to the conditions that a 
mosquito can reproduce, how many eggs are layed, how many hatch into larvae?).

\noindent We focus our studies in the state of Sonora, located at the Northwestern part of Mexico. An important case was discussed in (\cite{Ernst2016}), in which two populations, Hermosillo and Nogales, where the subject of study.

\noindent Several authors have considered the effects of temperature on mosquito dynamics and there has been also different approaches to understand dengue dynamics. Nonlinear ODE systems have been proposed. For example, Esteva and Yang (cita) consider a model based on epidemiological data to understand the effects of temperature on the intensity of dengue outbreaks. In (Beck-Johnson 2013) (to read carefully),...they do similar as EstevaYang. On (Lambrechts 2011) and (Liu-Helmersson 2014), there are studies on different parameters involved in the vectorial capacity as a function of temperature and diurnal temperature range (DTR). In (Marinho2016.pdf) Marinho et al studied the effect of temperature of the life cycle of the mosquito.


\noindent However as discussed in \cite{Ernst2016}), temperature also plays an important role in the survival of the adult female mosquito. In their study, they considered two different populations from northwest Mexico, Nogales and Hermosillo. In their work, it was found that larval and pupal abundance was greater in Nogales, but dengue cases, where much higher in the city of Hermosillo. The importance of temperature was that adult mosquitoes were able to survive longer in Hermosillo and therefore, being able to transmit the virus.

\noindent From this study, it is clear that not only temperature has played an important role in dengue presence in the medium but also, there is a socioeconomical factor that has to be taken into account. In case of having the ideal temperature for having infected mosquitoes, but there are no favourable socio-economical conditions for propagation, then a possible outbreak might not occur but only might be an endemic situation.

