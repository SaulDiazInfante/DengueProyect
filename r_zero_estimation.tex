%!TEX root = ./main.tex
We suppose that the number of cases of classic and hemorrhagic
dengue are observed at time points $t_{1}, \dots , t_{n}$.
Here we assume that these processes, denoted by $X_{t}$ and
$Y_{t}$ respectively, follow a Poisson distribution with mean
$\lambda_{h}\left(t\right)=Y_{-1}^{[h]}$ and
$\lambda_{c}\left(t\right)=Z$, where
    \begin{equation}
	\frac{dZ}{dt} =
		p 
		\left(
			I_1+ I_2 + Y_{-1}^{[c]}
		\right).
\end{equation} % Montoya
    \input{estimation.tex}    % Montoya
In our case the vector of parameters of the ordinary differential
equations model is $\phi=\left(\phi_{1},\phi_{2}\right)$, where
$\phi_{1}=\left(\beta_{H},\beta_{M}\right)$ is regarded as unknown and
$\phi_{2}=\left(\alpha_{c},\alpha_{h},b,\mu_{H},\mu_{M},\sigma,\theta,p\right)$
is known in advance. We write $\lambda_{h}\left(t\right)$ and
$\lambda_{c}\left(t\right)$ as $\lambda_{h}\left(t;\phi_{1}\right)$
and $\lambda_{c}\left(t;\phi_{1}\right)$ to emphasize this fact.

We use the likelihood approach to estimate the vector
parameter  $\psi_{1}$ based on the observed samples
$\vec{x}=\left(x_{t_1}, \dots , x_{t_n} \right)$ and
$\vec{y}=\left(y_{t_1} \dots , y_{t_n} \right)$. The resulting
likelihood function is thus

\begin{equation}
    \begin{split}
        L \left( \phi_{1} \right)
         &= \prod_{i = 1} ^ {n} 
         \left \{ 
             \frac{1}{x_{t_i}!}
                 \left[\lambda_{h}
                     \left(t_{i};\phi_{1}\right)
                 \right] ^ {x_{t_i}} \exp
                 \left[
                     \lambda_{h} \left(t_{i}; \phi_{1} \right)
                 \right] 
         \right. 
        \left.
             \frac{1}{y_{t_i}!}
             \left[\lambda_{c}
                 \left(t_{i};\phi_{1}\right)
             \right] ^ {y_{t_i}}
             \exp
             \left[
                 \lambda_{c}
                 \left(t_{i};\phi_{1}\right)
             \right]
         \right\}.\label{LikFun}
    \end{split}
\end{equation}
The maximum likelihood estimate (MLE) of $\phi_{1}$ is that value 
of $\phi_{1}$ that maximizes $L\left(\phi_{1}\right)$ in 
\eqref{LikFun}. We denote the MLE of $\phi_{1}$ as $\hat{\phi}_{1}$.

    We now consider profile-likelihood inference based on (1) for
estimating the parameters of interest ($R_{01}$, $R_{02}$, and
$\mathcal{R}_{0}$). Here we assume without loss of generality that
$\phi_{1}=\left(\beta_{H},\beta_{M}\right)$ can be rewritten as
$\phi_{1}=\left(\gamma,\eta\right)$, where $\gamma$ is a scalar
parameter of interest and $\eta$ is a scalar nuisance parameter. For
example, we may only be interested in $R_{01}$. In this case, we can
rewrite the parameter $\beta_{M}$ as a function of the parameters
$R_{01}$ and $\beta_{H}$, 
\begin{equation}
    \beta_{M}=C\frac{R_{01}}{\beta_{H}}, \nonumber
\end{equation}
where
\begin{equation}
    C=\left[\left(\frac{N_H - N_{S_{-1}}}{\alpha_c + \mu_H}
        +\frac{(1- \theta ) \sigma N_{S_{-1}}}{ \alpha_c + \mu_H}\right)
        \left(\frac{b^2\Lambda_M}{\mu_M ^ 2  N_H ^ 2}\right)\right]^{-1} \nonumber
\end{equation}
    Thus, we reparametrize the model in terms of 
$\phi_{1}=\left(\gamma,\eta\right)=\left(R_{01},\beta_{H}\right)$, 
where $\gamma=R_{01}$ is the parameter of interest and 
$\eta=\beta_{H}$ is the nuisance parameter.

    The profile likelihood and its corresponding relative likelihood
function of $\gamma$, standardized to be one at the maximum of the
likelihood function, are 
\begin{equation}
    \begin{split}
        L_{\max}
            \left(
            \gamma\right)&=\max_{\eta}L\left(\phi_{1}=\left(\gamma,\eta \right)
        \right),  \nonumber \\
        R_{\max}
        \left(
            \gamma\right)   
            &=  
                \frac{
                    L_{\max} \left( \gamma \right)
                }{
                    \displaystyle \max_{\phi_{1}
                } L
                \left( 
                    \phi_{1}
                \right)},
    \end{split}
\end{equation}
where $L\left(\cdot\right)$ is the likelihood function given in
(\ref{LikFun}). In particular, the relative profile likelihood varies between
0 and 1 and ranks all possible $\gamma$ values based only on the observed
samples $\vec{x}=\left(x_{t_1}, \dots , x_{t_n} \right)$ and
$\vec{y}=\left(y_{t_1} \dots , y_{t_n} \right)$. Thus, a graph of
$R_{\max}\left(\gamma\right)$ allows to distinguish plausible and implausible
values for $\gamma$.

A level $\omega$ profile likelihood region (commonly an interval) for $\gamma$ is given by
\begin{equation}
    \left\{\gamma: R_{\max}\left(\gamma\right) \geq \omega \right\}, \nonumber
\end{equation}
where $0 \leq \omega \leq 1$. We can assign a confidence level to the
profile likelihood region of $\gamma$  considering the asymptotical
behavior of the likelihood ratio statistic
$D=-2\ln{R_{\max}\left(\gamma_{0}\right)}$. This is an asymptotic
pivotal quantity having a Chi-squared distribution with one degree of
freedom. Thus, approximate confidence levels of $99\%$, $95\%$ and
$90\%$ can be ascribed to profile likelihood regions at $\omega$ =
0.036, 0.146, and 0.25, respectively.
\section{Results}
        Results should be clear and concise.
        \paragraph{Modeling Results (Saúl-Daniel)}

\paragraph{Data analysis (Saúl-Montoya)}
        \paragraph{$R_0$ and parameter inference (Montoya)}

\section{Discussion}
        This should explore the significance of the results of 
    the work, not repeat them. A combined results and 
    Discussion section is often appropriate. Avoid extensive 
    citations and discussion of published literature.

        In our results we obtained $R_{0c}>1$ and  $R_{0h}<1$. This means that 
    for this outbreak, the presence of DHF cases, cannot trigger on its own new 
    DHF cases and in general, $R_{0h}<1$ would imply an exponential decay on 
    the number of infected individuals. However, the small DHF outbreak arises 
    despite the value of $R_{0h}$ as there is an increase in infected 
    mosquitoes of the serotype 2 due to the presence of the $S$ individuals, 
    which initially is close to $N_S$. Therefore, DHF dynamics is a consequence 
    of the intensity of the outbreak of DF, given by serotype 2. 
%

        In our results we obtained $R_{0c}>1$ and  $R_{0h}<1$. This means
    that for this outbreak, the presence of DHF cases, cannot trigger
    on its own new DHF cases and in general, $R_{0h}<1$ would imply an
    exponential decay on the number of infected individuals. However,
    the small DHF outbreak arises despite the value of $R_{0h}$ as
    there is an increase in infected mosquitoes of the serotype 2 due
    to the presence of the $S$ individuals, which initially is close
    to $N_S$. Therefore, DHF dynamics is a consequence of the
    intensity of the outbreak of DF, given by serotype 2.

\noindent Our model was written following two purposes. To describe the evolution of DCF and DHF
cases in Hermosillo for the 2010 outbreak and to evaluate the hypothesis that secondary infections
with DENV-2 serotype were the main responsible for DHF cases. In order to meet our needs, our model is useful for one year dynamics only.