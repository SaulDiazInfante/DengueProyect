%!TEX root = ./main.tex
\section*{Important Hypothesis (Daniel-Saúl)}
    \paragraph{Permanent immunity}
    Accordingly to \cite{WHO}, Dengue infection caused by a DEN-i serotype induces long-life immunity to reinfection for this strain. Also,  a recovered individual previously infected with DEN-i Dengue serotype, acquires partial immunity to a different serotype for about a period of two years \cite{Reich2013}. As our study focuses on a single year dynamics, we assume that a susceptible individual who has never get infected before, could obtain Dengue by any of serotypes DENV-1, or DENV-2, and then becoming recovered to any strain for the rest of the year. 
    
    %
    \paragraph{ADE hypothesis}
    The processes and factors that produce \ac{DHF} are still unclear. Different factors have
    been observed to be responsible for DHF \cite{Martina2009}. However, one of the most predominant hypothesis claims that reinfection with a different serotype enhances the probability of developing plasma vascular permeability---the \ac{ADE} hypothesis \citep[see, e.g.][p. 295]{Halstead1992}, \cite{Guzman2013}. We consider in our formulation the ADE hypothesis, that 
    is, only a fraction of the second reinfection with serotype \ac{DENV-2} develops vascular leaking. Additionally, a second consequence of the ADE hyphotesis, is that the susceptibility of acquiring dengue a second time, is increased \cite{Recker2009}. \\
    %But there also exist studies that report first infection DHF cases \cite{Debast1993}.
%
    \paragraph{\ac{DENV-2} report circulation in Hermosillo}
    According to \cite{Vazquez2011} and \cite{Reyes2017}, in year 2010 only DENV-1 was confirmed to be present in the state of Sonora. In contrast, even there is currently presence of DENV-2 in the state, due to limited serotyping of cases it is not clear when DENV-2 began circulating the state \cite{Reyes2017}. In order to argue that reinfection with DENV-2 might have been responsible for DHF, we follow previous studies, such as the work by \cite{Gomez2014} and \cite{Vazquez2011}. In \cite{Gomez2014}, the authors analyze the dengue situation in Mexico for the period 2000 to 2011, and argue that the increase of DHF in 2001 in Yucat\'an was linked to the introduction of the DENV-2 strain. 
    On the other hand DENV-2 was reported to be present in Sinaloa and Baja California Sur in 2009 \cite{Vazquez2011}, which are neighbor states of Sonora. These studies support our hypothesis that DHF in 2010 in Hermosillo, was due to the introduction of DENV-2 into the state of Sonora. 
    
    \paragraph{Asymptomatic and reported cases} In order to estimate the solution that best fit the 
    evolution of the number of reported cases, we utilize a fraction $p$ of the individuals that our model identifies as DCF cases. Such fraction is based on two factors (i) The proportion of asymptomatic cases respect to the total infected individuals, which ranges from the $75\%$ to the $80\%$ (\cite{Bosch2018}, \cite{Reiter2010}) and (ii) from the symptomatic cases, only a fraction of them
     go to the hospital, and the test for DEN-V is taken. 
    
%    On the other hand, reported cases represent only a fraction of the infected individuals obtained by the model. Then, our data is compared to a fraction of individuals that belong to the first and second time infected classes.
%    We assume that the $X\%$ (pending) of the \ac{DCF} are asymptomatic, whereas for 
%    \ac{DHF}  all the cases are reported. 
    
    
%    Therefore, the class $Y_{-1}^{[h]}$ accounts for
%    all the \ac{DHF} cases, whereas a fraction $p=0.05$ of the sum 
%    $I_1+ I_2 + Y_{-1}^{[c]}$ represent the confirmed cases of \ac{DCF}. (75\% is the amount of asymptomatic according to literature)

    \tinytodo{Incluir citas de los 
        porcentages de asintomáticos,
        http://www.who.int/%
        en/news-room/fact-sheets/%
        detail/dengue-and-severe-dengue %
        says that about% 
        75$\%$ is asymptomatic}

    \todo{chastel2012.pdf}
    
    \paragraph{Homogeneity about the early outbreak stage}
In this model    
    

    \paragraph{Vector transmission dynamics}
        Define the infection forces as
    \begin{equation}
        \begin{aligned}
            A_{I_1} &=
                \frac{\beta_Mb}{N_H} I_1, \qquad
            A_{I_2}=
                \frac{\beta_Mb}{N_H} I_2,
        \\
            A_{Y_{-1}^{[h]}}&=
            \frac{\beta_Mb}{N_H} Y_{-1} ^{[h]}, \qquad
            A_{Y_{-1}^{[c]}}=
                \frac{\beta_Mb}{N_H} Y_{-1}^{[c]},
        \\
            B_{M_1} &= 
                \frac{\beta_Hb}{N_H}M_1, \qquad
            B_{M_2}=
                \frac{\beta_Hb}{N_H}M_2 ~.
        \end{aligned}
    \end{equation}

    Define
    $$
        A_{\bullet}:=
            A_{I_1} + A_{I_2} + A_{Y_{-1}^{[h]}} + 
            A_{Y_{-1}^{[c]}}
    $$
    as the total human infection force, that is, 
    the sum of all
    human contributions to the vector infection. 

    Then we describe the mosquito disease dynamics 
    by 
    \begin{equation}
        \begin{aligned}
            \\
            \frac{dM_S}{dt}&=
                \Lambda_M
                - A_{\bullet} M_S
                - \mu_M M_S
            \\
            \frac{dM_1}{dt}&=
                A_{I_1}  M_S - \mu_M M_1
            \\
            \frac{dM_2}{dt} &=
                \left(
                    A_{I_2}+A_{Y_{-1}^{[h]}}+A_{Y_{-1}^{[c]}}
                \right) 
                M_S-\mu_M M_2
        \end{aligned}
    \end{equation}
    Here $M_S$, is the vector susceptible class and
    $M_1$, $M_2$ respectively denotes the vector 
    Infected classes with \ac{DENV-1}
    and \ac{DENV-2}. 
    
    
    
    \paragraph{Host disease dynamics}
        Susceptible individuals ($S$) become infected for the first time
        with DENV-1 or DENV-2 after a successful mosquito bite and move
        to classes $I_1$ and $I_2$, respectively. Individuals in these classes can 
        be symptomatic or asymptomatic. From \cite{Bosch2018}, we know that asymptomatic 
        individuals are able to transmit the disease with an $80\%$ of effectiveness, compared 
        to symptomatic cases. Therefore, our model, makes no distinction between symptomatic and asymptomatic individuals for the disease dynamics.  
        After $\alpha_c^{-1}$ time units,. infected individualsFrom here, they remain
        in the infected class for $1/\alpha_c$ time units, after which,
        move to a recovered class $R_S$.  
        As we are interested in a one
        year dynamics, for the rest of the epidemic they become immune to
        any serotype. A second class of susceptible individuals $S_{-1}$,
        consist on those who acquired DENV-1 in previous years and in the
        current year are susceptible only to DENV-2. Such individuals
        become infected with DENV-2 when exposed to infected mosquitoes
        with that serotype. It is worth mentioning that asymptomatic individuals are 
        able to transmit the disease with an $80\%$ of effectiveness, compared 
        to symptomatic cases \cite{Bosch2018}. Therefore, our model, makes no 
        distinction between symptomatic and asymptomatic individuals for the disease dynamics. 
    
        
        In \cite{OhAinle2011} and
        \cite{Sangkawibha1984} it was observed that a more severe version
        of dengue occurs (might occur?) when an individual acquires dengue
        for a second time, and this happens to be DENV-2. Based on this
        assumption, an individual from $S_{-1}$ moves to $Y_{-1}^{[c]}$ or
        $Y_{-1}^{[h]}$, if the infection leads to DCF or DHF, respectively.
        Finally, these infected individuals move to the recovered class
        $R_{S_{-1}}$ at rates $\alpha_c$ and $\alpha_h$, respectively. For
        our model, $\mu_H$ is the human death rate; $b$ is the number of
        bites per week per mosquito and $\beta_H$ is the effectiveness of
        the bite. From the current hypothesis our model is given by    

%Assuming sufficient density of vectors, we 
%describe the \ac{DCF} and \ac{DHF},
%using a cross infection mechanism similar to the 
%reported in \cite{Feng1997a}. 
%Our version allows a hemorrhagic description of 
%the Infected human population 
%with serotype $i$---see 
%\Cref{tbl:variable_description,tbl:parameter_description}
%to variable and parameter description. The symbol 
%$I_{i}$ represents the human infected 
%pupulation with serotype $i$,
%which never was infected before, while 
%$Y_{-i}^{[\star]}$ denotes the 
%number of humans which are immune to serotype 
%$i$ at the time of the 
%present outbreak and develops dengue of type 
%$\star$ (DCF or DHF).
%Then $S_{-i}$ are the individuals who were 
%infected with serotype $i$ in the 
%previous outbreak, and at the current time, are 
%susceptible only to a serotype different of $i$.
%Dengue cross infection.

\begin{equation}\label{eqn:model_two_strains1}
    \begin{aligned}
        \frac{dS}{dt} &=
            \mu_HN_S - (B_{M_1} + B_{M_2}) S
            -\mu_H S
        \\
        \frac{dI_1}{dt} &=
            B_{M_1} S
            -(\alpha_c + \mu_H) I_1
        \\
        \frac{dI_2}{dt} &=
            B_{M_2} S
            -(\alpha_c + \mu_H)I_2
        \\
        \frac{dR_S}{dt}&=\alpha_c (I_1+I_2)-\mu_H R_S
        \\
        \frac{dS_{-1}}{dt} &=
            \mu_HN_{S_{-1}}- \sigma B_{M_2} S_{-1}-\mu_H S_{-1}
        \\
        \frac{dY_{-1} ^{[c]} }{dt} &=
            (1 - \theta) \sigma B_{M_2} S_{-1}
            -(\alpha_c + \mu_H) Y_{-1} ^ {[c]}
        \\
        \frac{dY_{-1}^{[h]}}{dt} &=
            \theta \sigma B_{M_2} S_{-1}
            -(\alpha_h + \mu_H)Y_{-1} ^{[h]} 
        \\
        \frac{dR_{S_{-1}}}{dt} &= 
            \alpha_c Y_{-1} ^{[c]}
            + \alpha_h Y_{-1} ^ {[h]} - \mu_H R
    \end{aligned}
\end{equation}
    Here, we take $N_H=N_S+N_{S_{-1}}$ as the total number of individuals.
    For our formulation $N_H, N_S$ and $N_{S_{-1}}$ remain constant. $N_S$
    is the total number of individuals that are involved in the first
    infection dynamics ($N_S = S +I_1+I_2+R_S$). On the other hand
    $N_{S_{-1}}$ is the total number of individuals involved in the
    reinfection dynamics ($N_{S_{-1}}=S_{-1}+Y_1^{[c]}+
    Y_1^{[h]}+R_{S_1}$). Also, the recovered individuals in both classes
    can be considered as a single recovered class $R=R_S+R_{S_{-1}}$ as
    our dynamics are taken only for one year. Then, our equations become

\begin{equation}\label{eqn:model_two_strains2}
    \begin{aligned}
        \frac{dS}{dt} &=
            \mu_H N_S - (B_{M_1} + B_{M_2}) S
            -\mu_H S
        \\
        \frac{dI_1}{dt} &=
            B_{M_1} S
            -(\alpha_c + \mu_H) I_1
        \\
        \frac{dI_2}{dt} &=
            B_{M_2} S
            -(\alpha_c + \mu_H)I_2
        \\
        \frac{dS_{-1}}{dt} &=
            \mu_HN_{S_{-1}}- \sigma B_{M_2} S_{-1}-\mu_H S_{-1}
        \\
        \frac{dY_{-1} ^{[c]} }{dt} &=
            (1 - \theta) \sigma B_{M_2} S_{-1}
            -(\alpha_c + \mu_H) Y_{-1} ^ {[c]}
        \\
        \frac{dY_{-1}^{[h]}}{dt} &=
            \theta \sigma B_{M_2} S_{-1}
            -(\alpha_h + \mu_H)Y_{-1} ^{[h]} 
        \\
        \frac{dR}{dt} &= 
            \alpha_c 
                \left(
                    I_1 + I_2 + Y_{-1} ^{[c]}
                \right)
            + \alpha_h Y_{-1} ^ {[h]} - \mu_H R
    \end{aligned}
\end{equation}


\todo{Fix $\Lambda _{\cdot} = N$. }
\todo{Include information about the 2 strains}
%!TEX root = ./main.tex

\begin{figure}[htb]
	\centering
	\includegraphics[width=\linewidth]{disiase_flow.pdf}
	\caption{Flow diagram of model \eqref{eqn:model_two_strains}}.
	\label{fig:disiaseflow}
\end{figure}
%
\begin{table*}[h!]
	\begin{center}
		\begin{tabular}{rl}
			\toprule
			Symbol		&	\multicolumn{1}{c}{Meaning}
			\\
			\midrule
			$M_S$
				& Number of susceptible mosquitoes.
			\\
			$M_1$, $M_2$
				&
				 Number of infected mosquitoes with virus
				\\
				& 
				serotype \ac{DENV-1} or \ac{DENV-2}.
			\\
			$S$
				&
				Susceptible host population which, 
				\\
				& never has acquired dengue.
			\\
			$S_{-1}$
			&
				Susceptible host population 
				which is immune to
			\\
			&
				serotype $1$.
			\\
			$I_1$, $I_2$
			&
				First time infected host population by 
			\\
				& serotype $1$ and $2$, respectively.
			\\
				$Y_{-1}^{[h]}$,
				$Y_{-1}^{[c]}$
				&
				Second time infected host population with 
				\\
				&
				serotype 2, with \ac{DHF} and \ac{DCF}, 
                \\
                &
                respectively.
			\\
		\bottomrule
		\end{tabular}
	\end{center}
	\caption{
		Meaning of variables. 
		Here we omit the explicit dependence of
		time.
	}\label{tbl:variable_description}
\end{table*}

%
%
\paragraph{Basic reproductive number}
    The disease free equilibrium results
$$
    FDE=
    \left(
        \frac{\Lambda_M}{\mu_M},
        0,
        0,
        N_H - N_{S_{-1}},
        0,
        N_{S_{-1}},
        0,
        0,
        0
    \right).
$$
\todo{get a relation for the initial grow phase parameter}
Using the next generation operator method 
reported as in \cite{Feng1997a}, we obtain
the basic reproductive number
%\begin{equation}
%   \begin{aligned}
%       \pi_R & :=
%           \frac{\beta_H \beta_M b^2 \Lambda_M}{
%               \mu_M ^ 2  N_H ^ 2 
%       }
%   \\
%       R_{01} & := 
%           \pi_R
%           \left(
%               \frac{N_H - N_{S_{-1}}}{ \alpha_c + \mu_H}
%               +
%               \frac{(1- \theta ) \sigma N_{S_{-1}}}{ \alpha_c + \mu_H}
%           \right)
%       \\
%       R_{02}& :=
%           \pi_R
%               \frac{
%                   \sigma \theta N_{S_{-1}}
%               }{\alpha_h + \mu_H},
%\qquad
%   \\
%   \mathcal{R}_0 & :=
%           \sqrt{ R_{01}+R_{02} }.
%   \end{aligned}
%\end{equation}
%
\begin{equation}
    \begin{aligned}
        R_{0c} & := \sqrt{
            \frac{\beta_MbN_M}{\mu_MN_H}
            \left(
                \frac{\beta_HbN_S}{ (\alpha_c + \mu_H)N_H}
                +
                \frac{\beta_Hb(1- \theta ) \sigma N_{S_{-1}}}{ (\alpha_c + \mu_H)N_H}
            \right)}
        \\
        R_{0h}& :=\sqrt{
            \left(\frac{\beta_MbN_M}{\mu_MN_H}\right)
                \left(\frac{
                    \beta_Hb \theta\sigma N_{S_{-1}}
                }{(\alpha_h + \mu_H)N_H}\right)}
%       \qquad
    \\
    \mathcal{R}_0 & :=
            \sqrt{ R_{0c}^2+R_{0h}^2 }.
    \end{aligned}
\end{equation}

        In this equation, $R_{0c}$ and $R_{0h}$, are the basic reproductive numbers 
    for classical and hemorrhagic dengue cases, respectively. From here, $R_0$ 
    provides a measure of how DF and DHF infected people influence the presence of 
    new dengue cases (Either DF or DHF). $R_{0h}$ measures the new hemorrhagic 
    cases that arise from one hemorrhagic infected individual in a population of 
    $N_{S_{-1}}$ susceptible to strain 2 individuals, meanwhile $R_{oc}$ provides a 
    measure of how many new individuals will obtain DC fever (DF?) from an individual 
    that has or has not have acquired dengue previously (from an individual that has either
    DF or DHF).

        Observe that this $R_0$ differs in some way to the traditional $R_0$ where two
    different serotypes are involved (\cite{Feng1997a} include citations of $R_0$ 
    for two serotypes).
    This follows from the idea that we are interested in classic and hemorrhagic
    cases rather than the predominance of a serotype.

% Changes test%