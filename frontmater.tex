\title{
		Modeling and $R_0$ estimation  of the 2010 Dengue Hemorrhagic Fever 
        Outbreak in Hermosillo Sonora.
%\tnoteref{t1}
}%,t2}}
%\tnotetext[t1]{
%	This work has been partially
%	supported by CONACYT project *****
%}
\author[conacyt-unison]{S. D\'{\i}az-Infante \corref{cor1}}
\ead{saul.diazinfante@unison.mx}
%
\author[unison]{J. A. Montoya Laos}
\ead{montoya@unison.mx}
%
\author[unison]{D. Olmos Liceaga}
\ead{daniel.olmos@unison.mx}
%
\author[colson]{P. A. Reyes Castro}
\ead{preyes@colson.edu.mx}

\address[conacyt-unison]{
	CONACYT-Universidad de Sonora,
	Departamento de Matem\'aticas, Universidad de Sonora\\
	Blvd. Rosales y Luis Encinas S/N, Col. Centro, Hermosillo, Sonora, C.P. 83000,
	M\'exico\\
}
%
\address[unison]{
	Departamento de Matem\'aticas, Universidad de Sonora\\
	Blvd. Rosales y Luis Encinas S/N, Col. Centro, Hermosillo, Sonora, C.P. 83000,
	M\'exico\\
}
%
\address[colson]{
	Centro de Estudios en Salud y Sociedad,
	El Colegio de Sonora, Avenida Obregón No. 54, Colonia Centro, C.P. 83000,
	Hermosillo, Sonora. México
}
%
\cortext[cor1]{Corresponding author}
\begin{abstract}
	We model and estimate the basic reproductive number of Dengue  and 
	Dengue Hemorrhagic Fever outbreak of the 2010 from Hermosillo Sonora.
	Our results suggest that serotype DENV-2 of the Dengue virus and the cross 
    infection risk enhancement hypothesis, could explain the incidence of
    Dengue Hemorrhagic Fever cases reported by Secretaria de Salud del Estado de 
    Sonora.
\end{abstract}
\begin{keyword}
	Differential Equations;
	Dengue Hemorrhagic Fever;
	Boot strap;
\end{keyword}
\journal{Acta Tropica}